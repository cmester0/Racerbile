
\documentclass[12pt]{article}

\usepackage{a4}
\usepackage[utf8]{inputenc}
\usepackage[danish]{babel}
\usepackage[T1]{fontenc}
\usepackage{amsmath}
\usepackage{makeidx}
\usepackage{float}


\setlength{\parindent}{0pt} %ingen indrykning

\usepackage{fancyhdr}
\pagestyle{fancy}
\fancyhf{}

\rhead{\bfseries\thepage}
\lhead{\bfseries Officielle Racerbileregler}
\addtolength{\headheight}{2.5pt}


\begin{document}

 \vspace*{\stretch{1}}
 \begin{flushright}
  {\Huge\bfseries Officielle Racerbileregler v. 2.0}\\
  \rule{\textwidth}{2pt}
  {\large\itshape Guldøl og terninger}\\
  \vspace*{\stretch{1}}
  {\bfseries af Lasse Letager Hansen\\[4ex]
Aarhus, Juni 2024\\}
 \end{flushright}
 \thispagestyle{empty}
 \vspace*{\stretch{1}}
 \clearpage

\newpage

\section*{Introduktion}

$\qquad$ Racerbile er et bilspil, det spilles med kapsler, terninger og øl. Det er med andre ord et drukspil.\\
Der er tre måder, hvorpå spillet kan spilles, som er uafhængigt af reglerne.
\begin{itemize}
	\item At forsøge at drikke mest selv
	\item At få de andre til at drikke mest
	\item Eller en kombination af de to
\end{itemize}

Der er under spillet visse ting, som skal siges højt. Disse står i anførselstegn.\\

Der er to typer fejl i spillet tekniske fejl og generelle fejl. Når der begås en fejl skal der drikkes en tår for hver fejl.\\

\subsection*{Herunder følger reglerne for Babyracerbile:}

\subsection*{Øl}

$\qquad$ Som standard drikkes der Guld Tuborg, når man spiller racerbile. I moderne udgaver er der 11 tårer i en Guld Tuborg på 33cl. Man kan dog vælge at drikke andet end Guld Tuborg. Er dit valg af øl samme styrke eller stærkere end Guld Tuborg er der 11 tårer pr. øl. Er det en lightøl(f.eks. Grøn Tuborg) er der 8 tårer på en 33cl flaske. Alle øvrige drikkevarer takseres som lightøl uanset styrke dog skal der minimum være 33cl i glasset/flasken.

$\qquad$ Vælger du at drikke noget der er stærkere end Guld Tuborg er der ingen rabat i antal tårer der drikkes. Prøver du, at undgå at drikke det korrekte antal tårer med henvisning til din drikkevares styrke, er det en generel fejl og der drikkes.

\subsection*{Banen}

$\qquad$ Banen består af hexagonale felter sat sammen, så de danner en lukket bane, ganske som en almindelig racerbane (og jo, en almindelig racerbane består også af hexagonale felter, man skal bare meget tæt på for at se dem).

\subsection*{Brikker}

$\qquad$ Hver spiller benytter en kapsel som brik. På kapselens inderside skrives en genkendelig markering. Denne side skal altid vende opad, også under kørselen, den eneste undtagelse herfra er Blokeringsfelter. At flytte med den forkerte side opad er en teknisk fejl(man kan ikke køre, hvis bilen ligger på hovedet). At flytte kapselen på højkant er en teknisk fejl(se særregel om at køre på to hjul)

\subsection*{Nøl}

$\qquad$ Forsinker man spillet kan man nøles. Det foregår ved, at de resterende spillere råber i kor: "Tre. To. En. Nøl." i et passende LANGSOMT tempo. Hvert nøl tæller, som en generel fejl mod spilleren, som bliver nølet. Det er en teknisk fejl, at nøle sig selv.

$\qquad$ Ufuldstændig liste af eksempler: 

\begin{itemize}
	\item Du har ikke givet terningerne videre korrekt.
	\item Det er din tur, men du er ikke igang med at udføre din tur.
	\item Du er ved at hente øl.
	\item Du er på toilettet.
	\item Du sidder og snakker midt i din tur.
\end{itemize}


\subsection*{Starten på Spillet}

$\qquad$ Som udgangspunkt i Babyracerbile slår man om, hvem der starter. Højeste slag vinder, næsthøjeste slag får næste tur. Er der spillere, som slår ens, slår de om indtil det er afgjort hvem der ligger forrest af dem. Derefter placerer spillerne sig i rækkefølge omkring bordet. Der spilles i positiv omløbsretning. Den originale regel om tidskørsel findes i et senere afsnit.

$\qquad$ Alternativt stater spilleren med højeste slag, og derefter kører man i positiv omløbsretning.

$\qquad$ Kapslerne placeres derefter på banen ved startlinien. Spilleren, som starter, placerer sin kapsel på det felt, som er tættest på startlinien uden den krydses. Derefter placerer de resterende spillere deres kapsel i rækkefølge bag poleposition.


\subsection*{Terninger}

$\qquad$ Spillet spilles med 1 terningbæger og 3 sekssidede terninger. Hver terning repræsenterer et gear og der slås med et antal terninger svarende til det gear man er i: første, andet og tredje gear.

$\qquad$ Man starter altid spillet i første gear(1 terning). Dette skal ikke annonceres. Der kan ikke geares op i første tur.

$\qquad$ Opgearing: Ønsker man at slå med flere terninger foretager man en opgearing, man kan højest foretage en opgearing pr. tur. Opgearingen skal annonceres i starten af turen: "Jeg gearer op til andet/tredje gear". Glemmer man at annoncere opgearingen er det en generel fejl. Der er altså to mulige opgearinger: Fra første til andet gear eller fra andet til tredje gear.

$\qquad$ Nedgearing: Ønsker man at slå med færre terninger foretager man en nedgearing, man kan højest foretage en nedgearing pr. tur. Nedgearingen skal annonceres i starten af turen: "Jeg gearer ned til første/andet gear". Glemmer man at annoncere nedgearingen er det en generel fejl. Der er altså to mulige nedgearinger: Fra tredje til andet gear eller fra andet til første gear. At geare ned er naturligvis en generel fejl.

$\qquad$ En etter er en koblingsfejl, det er en teknisk fejl og der drikkes en tår for hver etter man slår, med mindre man slår 3 ettere(trippel koblingsfehler), hvor man har smadret gearkassen og bunder resten af sin øl mens den bliver skiftet.

$\qquad$ Femmere og seksere slås om, for SÅ hurtigt kan man SLET ikke køre. Hvis man 3 gange slår en femmer eller en sekser er det en teknisk fejl. Hver efterfølgende femmer eller sekser er en teknisk fejl. Dvs. de første 3 femmere eller seksere er 1 teknisk fejl. Den 4. femmer eller sekser er 1 yderligere teknisk fejl.

$\qquad$ Slår man 7, 8 eller 9 kører man hasarderet kørsel. Hjulene hviner hver gang man drejer dvs. ændrer kørselsretning, man skal sige "hvin". Der er naturligvis slitage på hjulene, når man kører hasarderet kørsel. Derfor er der en teknisk fejl for hvert "hvin". Glemmer man at annoncere et hvin er det en generel fejl.

$\qquad$  Slår man 10, 11 eller 12 kan man ikke dreje eller bremse og risikerer derfor at køre af banen. Hvis man kører af banen drikker man 1 tår for hvert felt man ikke rykker på banen(teknisk fejl). For at komme på banen igen skal man i første gear(tvungen nedgearing) og turen efter bakker man ind på banen til det felt hvor man kørte af banen.

\subsection*{Felter}

Der er tre typer felter i racerbile.

\begin{itemize}
	\item De almindelige felter
	\item Nedgearingsfelter
	\item Blokeringsfelter
	\item Og midtv...
\end{itemize}

Der er FIRE typer felter i racerbile!

\subsubsection*{De almindelige felter}

Er typisk gule. Der er ingen særlige regler for de almindelige felter.

\subsubsection*{Nedgearingsfelter}

$\qquad$ Er typisk markeret med mørkeblå farve. Ved start på dette felt er der tvungen nedgearing, dvs. man mister muligheden for at beslutte om man vil gear op. Hvis man ikke er i første gear, skal man derfor gå et gear ned, hvilket selfølgelig er en general fejl.

\subsubsection*{Blokeringsfelter}

$\qquad$ Er typisk et rødt felt, evt. markeret yderligere med en stjerne. I dette felt er der kun plads til en spiller. Når man lander på feltet, skal man vende sin kapsel for at signalere, at man blokerer feltet. Så længe der blokeres kan andre spillere ikke passere igennem feltet, men er tvunget til at standse umiddelbart inden. Hvis en spiller skulle have landet på eller passeret igennem et blokeret felt ophæves blokeringen. Det foregår på følgende måde: Spilleren bringer sin kapsel til standsning feltet inden blokeringsfeltet, tager modstanderens kapsel og banker den en gang mod bordet for at markere, at der har været en kollision, derefter lægges kapselen tilbage på plads men nu vendt normalt.

\subsubsection*{Og midtvejsfeltet}
$\qquad$ Har forskellige markeringer. Enten en streg igennem feltet eller på Stalinbanen(tm) markeret med hammer og segl.
Når en spiller passerer midtvejsfeltet, man skal altså helt forbi feltet/stregen, annoncerer spilleren "midtvejsskål", hvorefter alle drikker 1 tår. Det er naturligvis en generel fejl at glemme at annoncere midtvejsskål eller at undlade at drikke for midtvejsskål, når den annonceres. Så er du f.eks. på WC, når der annonceres midtvejsskål drikker du 1 tår for midtvejsskål og 1 tår for den generelle fejl.

\subsubsection*{Generelle regler for felter}

$\qquad$ Banen kan dele sig i to visse steder. Her er reglen, at den FØRSTE spiller, som kommer til det pågældende sted vælger retning og samtlige spillere skal følge den samme retning. Det er naturligvis en teknisk fejl per felt man rykker ad den forkerte retning. Man skal derfor rykke tilbage og kører rigtigt!\\

$\qquad$ Er der en pil på banen angiver pilen den eneste korrekte vej igennem.\\

$\qquad$ Blokering: Der er maksimalt plads til to biler/kapsler på et felt. Når der er to kapsler på et felt, er det blokeret og bagfra kommende spillere kan ikke køre igennem feltet, men er tvunget til at bremse eller køre uden om, hvis det er muligt. Der er ingen kollision ved blokering med to kapsler. Spiller nummer to på et felt løfter den anden kapsel og lægger sin egen nederst. Det tæller, som to tekniske fejl at lægge kapslerne i den forkerte rækkefølge(der er to kapsler, som er lagt forkert).

$\qquad$Det kan føre sjove situationer med sig: Spiller 1 ligger forrest. Spiller 2 lander på samme felt og lægger sin kapsel oven på. Bliver det ikke opdaget er spiller 1 nu teknisk set blevet overhalet og skal drikke og derefter flytte kapslerne så de ligger rigtigt. Det koster en teknisk fejl idet spiller 1 ikke har ret til at flytte på spiller 2's kapsel. Opdages det nu at spiller 2 har lavet fejl skal spiller 2 drikke en tår for hver af de to tekniske fejl, det er 2 tårer, som ikke blev drukket da de skulle, det kaster så 2 generelle fejl af sig.

$\qquad$ Samlet set: Spiller 1 drikker 1 tår for at ligge forrest(hvilket han gør uanset kapslernes placering).
Spiller 2 drikker 2 tårer for tekniske fejl og 2 tårer for generelle fejl.
Fun stuffs.


\subsection*{Turen}

\begin{itemize}
	\item Annoncer op- eller nedgearing, hvis du vælger eller bliver tvunget til en(kører af banen eller lander på nedgearingsfelt).
	\item Slå med det korrekte antal terninger i forhold til gear. Det er naturligvis en teknisk fejl at slå med forkert antal.
	\item Ryk det antal felter du har slået. Du kan vælge at rykke færre, dog skal du drikke 1 tår for hvert felt du ikke rykker(teknsk fejl) og du skal rykke mindst 1 felt.
	\item Du kan ikke dreje bilen mere end 60 grader fra felt til felt.
	\item Hvis du er blokeret kan du ikke rykke flere felter, og skal drikke 1 tår for hvert felt du ikke rykker(teknisk fejl).
	\item Annoncer hvad du drikker for, og drik alle de antal tårer du skal(tekniske/generelle fejl)
	\item Læg alle 3 terninger i bægeret og stil det ved den næste spiller.
	\item Begår man tekniske eller generelle fejl uden for ens egen tur drikkes der med det samme.
\end{itemize}


\subsection*{Teknisk Fejl}

En teknisk fejl er alt der har med direkte spilhandlinger at gøre og man drikker en tår for hver teknisk fejl.

\begin{itemize}
	\item At slå en koblingsfejl.
	\item At rykke den forkerte kapsel(1 fejl per felt).
	\item At flytte med den forkerte side opad(1 fejl per felt).
	\item At flytte på en andens kapsel, når det ikke er dikteret af reglerne. Man skal naturligvis lægge modstanderens kapsel tilbage på plads, som igen er en teknisk fejl.
	\item At 3 gange i en tur slå en femmer eller en sekser. Hver efterfølgende femmer eller sekser i samme tur er en extra teknisk fejl.
	\item At tabe en terning på gulvet eller hvis den triller ud over bordet.
	\item At slå en terning så den flytter på en kapsel.(en teknisk fejl pr kapsel der flyttes).
	\item At skulle hvine, når man drejer ved slag på 7, 8 eller 9(fordi man ødelægger dækkene).
	\item At rykke færre felter end det antal øjne man har slået.
	\item At slå med det forkerte antal terninger. En fejl per terning.
	\item At lægge kapsler på et blokeret felt i den forkerte rækkefølge.
	\item At køre den forkerte vej, når banen deler sig. En fejl per felt.
	\item At nøle sig selv.
	\item At slå en terning, så den lander skævt.
	\item At slå en terning, så den lander på andet end direkte på spilbordet(f.eks. på et kort eller en øletiket).
	\item At slå med terningerne, når det ikke er ens tur. Der kan stadig laves koblingsfejl.
        \item At slutter forrest eller starter bagerst.
\end{itemize}

\subsection*{Generel Fejl}

Er en fejl, som ikke har med direkte spilhandlinger at gøre. Man drikker en tår for hver generelle fejl.

\begin{itemize}
	\item At forklare en regel.
	\item At forklare en regel forkert.
	\item At glemme at annoncere en opgearing.
	\item At glemme at annoncere en nedgearing.
	\item At glemme at annoncere, at man starter i første gear efter tvungen nedgearing eller at være kørt af banen.
	\item At glemme at hvine, når man slår 7, 8 eller 9.(der er stadig en teknisk fejl for at skulle hvine, det koster altså 1 teknisk og 1 generel fejl at glemme at hvine).
	\item At glemme at annoncere midtvejsskål.
	\item At foretage en nedgearing.
	\item At undlade at drikke, når der annonceres midtvejsskål.
	\item At forsøge at undgå at drikke tårer ved at henvise til, at ens drikkevare er stærkere end Guld Tuborg.
	\item At blive nølet.
\end{itemize}

\newpage

\section*{Racerbile}

$\qquad$  Følgende regler adskiller Racerbile fra Babyracerbile:

\begin{itemize}

\item Du må ikke kun stoppe tidligt på et blokeringsfelt, dvs du skal rykke så langt some muligt.
 
\item Tidskørsel. For at afgøre, hvem der starter køres der tidskørsel. Det betyder, at hver spiller bunder en Guld Tuborg. Den der er hurtigst starter, derefter den næsthurtigste og så fremdeles. Spillerne skifter plads, så deres plads ved bordet matcher deres tidskørsel.

\item Slår man tre ettere smadrer man gearkassen. Efter en smadret gearkasse starter man i første gear. Der skal naturligvis drikkes for hver nedgearing. I starten af sin næste tur skal man annoncere: "Jeg starter i første gear". Det er naturligvis en generel fejl, hvis man glemmer at annoncere. %

\item Kører man af banen skal man bunde resten af sin øl inden man kan komme tilbage på banen.

\item Facing: Når du starter spillet kan du vælge en retning. Herefter peger din bil den retning du sidst kørte, som altså er direkte modsat det felt du kom fra. Det betyder, at du kan starte din tur med at skulle dreje. Det har betydning, når du slår 7, 8 eller 9, hvor du kan skulle hvine på det første felt du flytter. Eller når du slår 10, 11 eller 12, hvor du så ikke kan dreje og du vil køre af banen med det samme.

\end{itemize}

\subsection*{Særregler}

$\qquad$ Herunder er valgfrie regler, som ikke findes i normal Racerbile, men som er forsøgt brugt i spillet gennem tiderne.

\begin{itemize}

\item To hjul(kapselen flyttes på højkant). Kører man hasarderet kørsel (7, 8 eller 9) letter bilen og man kører på to hjul i svingene. Man skal altså ved hasarderet kørsel huske at hvine, og at skifte mellem at køre på to og fire hjul, to hjul når man drejer og fire hjul, når man kører ligeud. Det er naturligvis en teknisk fejl, hvis man ikke kører med det korrekte antal hjul på banen.

\item Er en spiller i starten af sin tur 12 felter eller mere foran nummer to, så har han kørt så stærkt, at han er løbet tør for benzin og han er nu tvunget til at bunde sin øl for at tanke op. Han mister sin tur og skal starte sin næste tur i første gear. Det er naturligvis en teknisk fejl, at være løbet tør for benzin, så efter optankningen drikker spilleren en tår.

\item Stalin. Der er både en Stalinbane og begrebet 'at køre Stalin', som i begyndelsen var den eneste måde man kørte på Stalinbanen. Ideologien er: Stalin gearer altid op. Stalin blokerer altid. Stalin kører altid af banen. Det giver i bund og grund sig selv. Ah, ja og Stalin ender altid i hegnet...

\item Flytter man sin kapsel med den forkerte side opad modtager, man alle de straftårer som givet ovenfor, og man skal dernæst tilbage til udgangspunktet og flytte sin kapsel forfra. Du kan altså drikke tårer for hvin to gange. Eller køre af banen to gange! :)

\end{itemize}

\newpage

\section*{Le Mans}

$\qquad$ Le Mans er en 24 timers version af Racerbile udtænkt engang i sluthalvfemserne af fysikernes rushold Kerne. Som i rigtig Le Mans er der 3 kørere på hvert hold. Spillet spilles på en særlig Racerbilebane lavet til formålet kaldet 'Le Mans-banen'. I modsætning til almindelig racerbile, hvor spillerne selv dømmer er der til Le Mans udnævnte dommere, som forventes at holde sig ædru nok til at udføre deres hverv.

$\qquad$ Af praktiske hensyn skal der være arrangører fra Cafeén?s side, der sørger for ædru-vagter, bartender(i tilfælde af publikum) og optællere. Disse skal ikke på nogen måde være en del af spillet. Der bør afsættes et afgrænset område for banen og et afgrænset soveområde for kørere og dommere, som tilskuere ikke må komme i nærheden af. De dele af holdet, der ikke kører kan frit gøre, som de har lyst, så længe de ikke forstyrrer de kørende spillere. Holdende har tidligere gjort brug af coaches. Det står holdende frit for at gøre det, men en eventuel coach skal holde sig fra banen, præcis som andre ikke-kørende elementer. Det er Cafeén?s beslutning, om de vil lade sig nøjes med, at hver kører køber en kasse GT.

$\qquad$ Der fremstilles drikkeglas med målestreger til hvert hold. Streger hver 3. cl således at 11 streger = 1 øl.

$\qquad$ Le Mans køres efter almindelige Racerbileregler med følgende ændringer:

\begin{itemize}

	\item Dommerne kan justere reglerne undervejs, hvis det skønnes nødvendigt. Hensyn kan være sikkerhed, spillets gennemførsel eller almindelig underholdning for tilskuere.

	\item Dommerne kan uddele straftårer efter forgodtbefindende.

	\item Der må køres i valgfri øl, men da det er en gentlemansport ses det gerne, at alle kører i Guld Tuborg. Se i øvrigt reglen ovenfor.

	\item Man kan ikke dømmes til at drikke mere end 42 tårer på en tur.

	\item 42 feltet. Lander en spiller på 42 feltet må vedkommende udnævne en modstander, som så skal drikke det samme, som en selv.

	\item Dunlopbroen. Vælter man Dunlopbroen bunder man en øl.

	\item Kaster man op ud over bordet koster det 5 omgange. Forsinkes spillet herved kan holdet diskvalificeres.

	\item Kaster man op på gulvet, så det kræver rengøring koster det 5 omgange.

	\item Man skal være i første gear, når man kører ind i pitten.

	\item Det er gratis at skifte kører i pitten.

	\item Vil man skifte kører uden for pitten, skal den indskiftede kører bunde en øl inden. Det skal annonceres inden turen starter og øllen skal være drukket inden turen kan starte. Der kan nøles.

	\item Der nøles 3 gange, og den fjerde gang rundes der op til en øl og turen gives videre. Hvis der næste tur ikke kan stilles en spillere, må dommerne vurdere om holdet er ude.

	\item Forsinkelse af spillet. Udfører man en handling uden for sin tur, som forsinker spillet, men ikke er dækket af ovenstående mister man sin næste tur og en omgang.

	\item Når der er gået 24 timer bliver der kaldt til sidste omgang. Spillet slutter når holdet, som fik dårligste tidskørsel har haft sin tur.

	\item Vinder af spillet er det hold, som har flyttet sig flest felter målt på antal omgange og dernæst placering på banen, skulle to hold ligge lige i antal omgange.

\end{itemize}


\end{document}
